\documentclass[a4paper]{article}
\usepackage[scale=0.75]{geometry}
\usepackage[utf8]{inputenc}
\usepackage[T1]{fontenc}
\usepackage{textcomp}
\usepackage[english]{babel}
\usepackage{amsmath, amssymb, bm}
\usepackage{physics}

\title{General feedback for EM Theory Problem Sheet 2}
\author{Kin Ian Lo}

\begin{document}
\maketitle

\section*{Question 1}
\label{sec:question_1}
This question is a variant of the classical EM problem in which a conducting rod sit on up of two parallel conducting rails, with a constant and uniform magnetic field.
\subsection*{(a)}
One common issue is that we are so used to writing down the flux as 
\[
\Phi = B A
,\] 
where $B$ is the (uniform) magnetic field strength and $A$ is the area. In this question, the magnetic field $\mathbf{B} = \beta_0 (x-x_0)^2 \mathbf{\hat{z}}$ is obviously not uniform. So when you write the above formula, it is ambiguous what $B$ refers to. Is it  $B(x)$ or $B(x_0)$ ?
Therefore, it is not correct to write down this formula and I deducted one mark for those who did.

To correctly calculate the magnetic flux through the area enclosed by the resistor, sliding arm and the rails, we have to do an integration:
\[
    \Phi(x) = \int_{x_R}^x dx' L |\mathbf{B}(x')|
.\] 
Note that the lower limit of the integral is $x_R$ which I defined it to be the position of the resistor on the x-axis.
Some of you put $x_0$ or $0$ in the lower limit. The question did not specify the position of the resistor and you should not assume  $x_R = x_0$ or  $x_R = 0$, but this does not affect the result after the time derivative. Therefore, although doing so is technically wrong but I did not deduct any marks for that. 
Substitute the given B-field gives
\[
    \Phi(x) = \int_{x_R}^x d' L \beta_0 (x-x_0)^2
.\] 
Evaluate the integral gives
\[
    \Phi(x) = L \beta_0 \frac{(x-x_0)^3}{3} - L \beta_0 \frac{(x_R-x_0)^3}{3}
.\] 
Notice that the second term is a constant. Taking the time derivative:
\[
    \frac{d\Phi(x(t))}{dt} =  L \beta_0 (x-x_0)^2 \frac{dx}{dt}
.\] 
Of course the above is not the most efficient way to get the required expression, because I have taken a derivative right after an integration. 
The most efficient way is to argue that $d\Phi = B(x) dA$ and we know that $A = L (x - x_R)$. So we have 
\[
    d\Phi = B(x) L dx
.\] 
We can then make the mathematically doggy move\footnote{We are physicists so we can live with that.} of divining both side with $dt$ to give the required expression:
\[
    \frac{d\phi}{dt} = B(x)L \frac{dx}{dt}
.\] 
\subsection*{(b)}
For the sub-question (b), the most common mistake was forgetting to argue for why the force opposes the direction of travel. This (unfortunately) worth one whole mark according to the marking scheme.

The most straight forward approach for this sub-question is to use Ohm's law\footnote{The question, however, did not explicitly say that the sliding arm and the rails have negligible resistance, which is quite unfortunately but one could have easily guessed this implicit assumption.
} to find out the current in the loop and use Lenz's law to argue that the current is clockwise.
And then use Lorentz force law to find the magnitude and direction of the force on the sliding arm. The direction of the force follows from right-hand-rule applied to the cross product in Lorentz force law.

A few of you used the fact that the dissipative power in the resistor is equal (in magnitude) to the rate of work done on the sliding arm. This is a fact but I would say this fact is not a common knowledge and should be avoid when doing problem sheets or exams, unless you have clearly explained the fact with reasoning using common knowledge.
For the benefits of the others who did not went down this path, I will show how this is done.

We know the dissipative power of resistor is 
\[
P_\text{resis} = \frac{V^2}{R}
.\] 
The power of work done on the sliding arm is 
\[
    P_\text{arm} = \frac{dx}{dt} F = v F 
,\] 
where $F$ is the magnitude of the Lorentz force on the arm.
Due to conservation of energy\footnote{Energy is conserved even though an external magnetic field is present. This is because a magnetic field does no work at all according to Lorentz force law.}, we required
\[
P_\text{arm} + P_\text{resis} = 0
.\] 
This gives an expression for $F$:
 \[
F = - v \frac{V^2}{R}
.\] 
One can then substitute the expression for induced potential $V = d\Phi/dt$ found in (a):
 \[
     F = - \frac{L^2 \beta_0^2}{R} (x-x_0)^4 v
.\] 

\subsection*{(c)}
For sub-question (c), as long as you know how to use the hint giving in the question then you are fine. The trick is to realise that 
\[
    v (x-x_0)^4 = \frac{dx}{dt} (x-x_0)^4 = \frac{1}{5} \frac{d(x-x_0)^5}{dt}
.\] 
So the equation of motion becomes
\[
    m \frac{dv(t)}{dt} = -\frac{L^2 \beta_0^2 }{R} \frac{1}{5} \frac{d(x-x_0)^5}{dt}
.\] 
Move everything to one side and combine the time derivative:
\[
    \frac{d}{dt} \left[ m v(t) + \frac{L^2 \beta_0^2}{5R} (x-x_0)^5\right] = 0
.\] 
It is obvious that the expression in the square bracket is a constant which can be found using the initial conditions:
\[
    (x, v)(t=0) = (x_0, v_0)
.\] 
One would say that the hint is not so straight forward to use and I agree. 
A few of you used a much quicker and much more intuitive method which I will show you here.
Starting with integration the equation of motion with respect to time:
\[
    m \int_0^t dt' \frac{d v}{dt}(t') = -\frac{L^2 \beta_0^2}{R} \int_0^t dt' v(t') (x(t')-x_0)^4
.\] 
By the fundamental theorem of calculus and recall that $v = dx/dt$:
\[
    m (v(t)-v_0) = -\frac{L^2 \beta_0^2}{R} \int_0^t dt' \frac{dx}{dt} (x(t')-x_0)^4
.\] 
Notice that we can change the variable of integration to $x$ now. We have 
 \[
     m(v(t)-v_0) = - \frac{L^2 \beta_0}{R} \int_{x_0}^x dx' (x'-x_0)^4
.\] 
Evaluating the trivial integral gives:
\[
m(v(t)-v_0) = - \frac{L^2 \beta_0^2}{R} \frac{1}{5} (x-x_0)^5
.\] 

Two of you decided to torture themselves\footnote{I torture myself as well by spending an hour verifying the by parts calculations} by integrating the equation of motion by parts. 
Consider the indefinite integral $I$:
 \[
     I = \int dt \frac{dx}{dt} (x-x_0)^4 
.\] 
Intuitively, the essence of integration by part is that you can ``move'' a derivative operation $d/dt$ to the other part of the integrand with a side-effect of a minus sign and with the consequence of having an extra term. Here, we use it to move $d/dt$ to act on $(x-x_0)^4$:
\[
    I = x(x-x_0)^4 - \int dt x \frac{d}{dt} (x-x_0)^4
.\] 
\[
    I = x(x-x_0)^4 - 4 \int dt x \frac{dx}{dt} (x-x_0)^3 
.\] 
Notice that $x \frac{dx}{dt} = \frac{1}{2} \frac{dx^2}{dt}$:
\[
    I = x(x-x_0)^4 - 2 \int dt \frac{dx^2}{dt} (x-x_0)^3 
.\]
Integration by part again:
\[
    I = x(x-x_0)^4 - 2 \left[ x^2 (x-x_0)^3 -  \int dt \frac{dx^3}{dt} (x-x_0)^2 \right]
.\] 
\[
    I = x(x-x_0)^4 - 2 x^2 (x-x_0)^3 + 2 \left[ x^3 (x-x_0)^2 - \frac{1}{2} \int dt \frac{dx^4}{dt} (x-x_0) \right]
.\] 
\[
    I = x(x-x_0)^4 - 2 x^2 (x-x_0)^3 + 2  x^3 (x-x_0)^2 - \left[ \int dt \frac{dx^4}{dt} (x-x_0) \right]
.\] 
\[
    I = x(x-x_0)^4 - 2 x^2 (x-x_0)^3 + 2  x^3 (x-x_0)^2 - \left[ x^4 (x-x_0) - \frac{1}{5}\int dt \frac{dx^5}{dt} \right]
.\]
\[
    I = x(x-x_0)^4 - 2 x^2 (x-x_0)^3 + 2  x^3 (x-x_0)^2 - x^4 (x-x_0) + \frac{1}{5}\int dt \frac{dx^5}{dt} 
.\] 
Eventually we arrive at the end of the journey of integration by parts and had to deal with the last integral:
\[
    \int dt \frac{dx^5}{dt} = \int d(x^5) = x^5 + C 
,\]
where $C$ is the constant of integration. Notice that we can do this change of variable form $t$ to  $x$ in the very first step and avoid all the subsequence integration by parts. 
Our equation of motion after integrating with respect to time now looks like
\[
    m v(t) = -\frac{L^2 \beta_0^2}{R} \left(x(x-x_0)^4 - 2 x^2 (x-x_0)^3 + 2  x^3 (x-x_0)^2 - x^4 (x-x_0) + \frac{1}{5} x^5 + C \right)
.\] 
Determine $C$ by imposing the initial conditions $(x,v)(t=0) = (x_0, v_0)$:
\[
    m v_0 = -\frac{L^2 \beta_0^2}{R} \left( \frac{1}{5} x_0^5 + C\right) 
.\] 
\[
C = - \frac{mRv_0}{L^2\beta_0^2} - \frac{1}{5} x_0^2
.\] 
Substitute $C$ back to the expression gives
 \[
     m (v(t)-v_0) = -\frac{L^2 \beta_0^2}{R} \left(  x(x-x_0)^4 - 2 x^2 (x-x_0)^3 + 2  x^3 (x-x_0)^2 - x^4 (x-x_0) + \frac{1}{5} x^5 -\frac{1}{5}x_0^5 \right)
.\] 
\[
     m (v(t)-v_0) = -\frac{L^2 \beta_0^2}{5R} \left(5 x(x-x_0)^4 - 10 x^2 (x-x_0)^3 + 10 x^3 (x-x_0)^2 - 5 x^4 (x-x_0) +  x^5 - x_0^5 \right)
.\] 
Observe the pattern $(5, 10, 10, 5, 1)$ in the coefficients. Rang a bell? Those are the binomial coefficients, or in other words, the fifth row in the pascal triangle\footnote{If you regard the top row as the zeroth row}. However, the term $-(x-x_0)^5$ is missing so we need to add it and subtract it:
\[
    (x-x_0)^5 + \left[- (x-x_0)^5 + 5 x(x-x_0)^4 - 10 x^2 (x-x_0)^3 + 10 x^3 (x-x_0)^2 - 5 x^4 (x-x_0) +  x^5 \right] - x_0^5 
.\] 
Which is just
\[
    (x-x_0)^5 + (x-(x-x_0))^5 - x_0^5 = (x-x_0)^5 + x_0^5 - x_0^5 = (x-x_0)^5
.\] 
So the polynomial is just $(x-x_0)^5$ and we recover the expression obtained above using the hint.
\section*{Question 2}
This question is about the magnetic field of the Earth (or any magnetic dipole really). Knowledge tested include integration with spherical coordinates and understanding of magnetic field lines.

The magnetic field of a dipole is 
 \[
     \bm{B} = - B_0 R_E^3 \left( 2 \frac{\cos\theta}{r^3} \mathbf{\hat{r}} + \frac{\sin\theta}{r^3} \hat{\bm{\theta}} \right) 
.\] 
\subsection*{(a)}
Not much can be said on this. You just have to know which component of the B-field to use. 
The surface consider here is the surface of the Earth but only limited to $\theta_1 \leq \theta \leq \theta_2$. The corresponding normal surface element is $d \bm{S} = dS \bm{\hat{r}}$. So the infinitesimal flux element is just 
\[
d\Phi_\text{S} = \bm{B} \cdot d \bm{S} = B_r dS 
.\] 
The surface element of a sphere of radius $R_E$ is $dS = R_E^2 \sin\theta d\theta d\phi$. Integrating the flux element from $\theta_1$ to $\theta_2$:
 \[
     \Phi_\text{S}(\theta_1, \theta_2) = \int_{\theta_1}^{\theta_2} \int_0^{2\pi} R_E^2 \sin\theta d\theta d\phi B_r(\theta, r=R_E) 
.\]
Evaluating the integral gives
\[
    |\Phi_\text{S}(\theta_1, \theta_2)| = 2 \pi B_0 R_E^2 \left| \cos^2\theta_1 - \cos^2\theta_2 \right| 
.\] 
\subsection*{(b)}
The surface in this sub-question is the equatorial plane ($\theta=\pi/2$) but limited to  $R_1 < r < R_2$. The corresponding normal surface element is $d \bm{S} = d S (-\hat{\bm{\theta}})$. So the infinitesimal flux element is just 
\[
d\Phi_\text{EQ} = \bm{B} \cdot d \bm{S} = B_\theta dS.
.\] 
The surface element of the equatorial plane is $dS = r dr d\phi$. Integrating the flux element from  $R_1$ to $R_2$:
\[
    \Phi_\text{EQ}(R_1, R_2) = \int_{R_1}^{R_2} \int_0^{2 \pi} r dr d\phi B_\theta(\theta=\pi/2, r)
.\]
Evaluating the integral gives
\[
\left|\Phi_\text{EQ}(R_1, R_2)\right| = 2 \pi B_0 R_E^2 \left| \frac{R_E}{R_1} - \frac{R_E}{R_2}\right|
.\] 
\subsection*{(c)}
This sub-question can be a tricky one if you fail to understand the question.
In general, the two fluxes are not always the same if you choose $\left( R_1, R_2, \theta_1, \theta_2 \right) $ arbitrarily, i.e.
\[
|\Phi_\text{S}(\theta_1, \theta_2)|  \neq \left|\Phi_\text{EQ}(R_1, R_2)\right| 
.\] 
When you enforce that $(\theta_i, R_i)$ are on the same magnetic field line, then the two fluxes the same. That is, if we impose the condition for $i \in \{1, 2\}$
\[
R_i = \frac{R_E}{\sin^2\theta_i}
,\]
then we have
\[
|\Phi_\text{S}(\theta_1, \theta_2)| = \left|\Phi_\text{EQ}\left(\frac{R_E}{\sin^2\theta_1}, \frac{R_E}{\sin^2\theta_2}\right)\right| 
.\] 
The above equality is what the question tended you to show. However, it is pretty easy to just guess that you have to use the magnetic field line equation ($R \sin^2\theta = R_E$) to somehow show the two fluxes, $\Phi_\text{S}$ and $\Phi_\text{EQ}$, are equal.
\subsection*{(d)}
Very few of you got this sub-question. The main point of this sub-question is to show that for any pair of $(\theta_1, \theta_2)$, we have 
\[
|\Phi_\text{S}(\theta_1, \theta_2)| = \left|\Phi_\text{EQ}\left(\frac{R_E}{\sin^2\theta_1}, \frac{R_E}{\sin^2\theta_2}\right)\right| 
.\] 
In other words, the question asks if we can determine the value of $\left|\Phi_\text{EQ}\left(\frac{R_E}{\sin^2\theta_1}, \frac{R_E}{\sin^2\theta_2}\right)\right| $, without actually doing the integration, given that we already knew the value of $|\Phi_\text{S}(\theta_1, \theta_2)| $. 

Many of you chose $(\theta_1, \theta_2)$ to be whatever you like them to be and your argument was solely based on your particular choice of $(\theta_1, \theta_2)$. 
I would like to stress again that you should not make any choices on $(\theta_1, \theta_2)$ (or equivalently, on $(R_1, R_2)$). You answer must be general enough to cover all possible $(\theta_1, \theta_2)$.

Many of you went on to show that the magnetic flux over some closed surface is zero. This is indeed the way to solve this problem but it is very important to choose the correct closed surface.
The most common choices for the closed surfaces are the surface of the Earth or a hemisphere of the Earth. Obviously these choices of surface are not general and therefore stand no chance in successfully solving the problem. I will give more reasons here for why they are not good choices of surface.

For the entire surface of the Earth, It is simply because your surface does not include any part of the equatorial plane. There is no way your Gauss law would include any information about $|\Phi_\text{EQ}|$.

Choosing a hemisphere (with the equatorial plane closing the surface) is also a bad idea even though it includes part of the equatorial plane. Remember we want to show that 
\[
|\Phi_\text{S}(\theta_1, \theta_2)| = \left|\Phi_\text{EQ}\left(\frac{R_E}{\sin^2\theta_1}, \frac{R_E}{\sin^2\theta_2}\right)\right| 
.\] 
But notice that $\frac{R_E}{\sin^2\theta} \geq R_E$, so the equatorial plane included in your surface will never appear in the equality the question asked you to prove. Therefore, you will not be able to answer the question with this choice of surface.

The correct choice of the surface is one that comprises of
\begin{itemize}
    \item ($S_\text{S}$) the surface used to calculate $\Phi_\text{S}(\theta_1, \theta_2)$
    \item ($S_\text{EQ}$) the surface used to calculate $\Phi_\text{EQ}\left(\frac{R_E}{\sin^2\theta_1}, \frac{R_E}{\sin^2\theta_2}\right) $
    \item ($S_\text{conn}$) the two sheets of surface, which both follow the B-field lines, such that $S_\text{S}$ and $S_\text{EQ}$ are joined together to form a closed surface. 
\end{itemize}
The reason for including $S_\text{S}$ and $S_\text{EQ}$ is pretty obvious. The connecting surface $S_\text{conn}$ is chosen follow B-field lines such that the magnetic flux $\Phi_\text{conn}$ through $S_\text{conn}$ is zero. 
Now Gauss law tells us that
\[
\Phi_\text{S}(\theta_1, \theta_2) + \Phi_\text{EQ}\left(\frac{R_E}{\sin^2\theta_1}, \frac{R_E}{\sin^2\theta_2}\right) + \Phi_\text{conn} = 0
.\] 
But $\Phi_\text{conn}$ is zero so we have the required result.
\end{document}
